\documentclass[12pt,preprint]{aastex}

\pdfoutput=1

\usepackage{color,hyperref}
\definecolor{linkcolor}{rgb}{0,0,0.5}
\hypersetup{colorlinks=true,linkcolor=linkcolor,citecolor=linkcolor,
            filecolor=linkcolor,urlcolor=linkcolor}
\usepackage{url}
\usepackage{amssymb,amsmath}
\usepackage{subfigure}
\usepackage{booktabs}

\usepackage{natbib}
\bibliographystyle{apj}


\newcommand{\project}[1]{\textsl{#1}}
\newcommand{\kepler}{\project{Kepler}}
\newcommand{\terra}{\project{TERRA}}
\newcommand{\license}{MIT License}

\newcommand{\paper}{\textsl{Article}}

\newcommand{\foreign}[1]{\emph{#1}}
\newcommand{\etal}{\foreign{et\,al.}}
\newcommand{\etc}{\foreign{etc.}}
\newcommand{\True}{\foreign{True}}
\newcommand{\Truth}{\foreign{Truth}}

\newcommand{\figref}[1]{\ref{fig:#1}}
\newcommand{\Fig}[1]{\figurename~\figref{#1}}
\newcommand{\fig}[1]{\Fig{#1}}
\newcommand{\figlabel}[1]{\label{fig:#1}}
\newcommand{\Tab}[1]{Table~\ref{tab:#1}}
\newcommand{\tab}[1]{\Tab{#1}}
\newcommand{\tablabel}[1]{\label{tab:#1}}
\newcommand{\Eq}[1]{Equation~(\ref{eq:#1})}
\newcommand{\eq}[1]{\Eq{#1}}
\newcommand{\eqalt}[1]{Equation~\ref{eq:#1}}
\newcommand{\eqlabel}[1]{\label{eq:#1}}
\newcommand{\sectionname}{Section}
\newcommand{\Sect}[1]{\sectionname~\ref{sect:#1}}
\newcommand{\sect}[1]{\Sect{#1}}
\newcommand{\sectalt}[1]{\ref{sect:#1}}
\newcommand{\App}[1]{Appendix~\ref{sect:#1}}
\newcommand{\app}[1]{\App{#1}}
\newcommand{\sectlabel}[1]{\label{sect:#1}}

\newcommand{\dd}{\ensuremath{\,\mathrm{d}}}
\newcommand{\bvec}[1]{\ensuremath{\boldsymbol{#1}}}
\newcommand{\densityunit}{{\ensuremath{\mathrm{nat}^{-2}}}}

% TO DOS
\newcommand{\todo}[3]{{\color{#2} \emph{#1} TODO: #3}}
\newcommand{\dfmtodo}[1]{\todo{DFM}{red}{#1}}
\newcommand{\hoggtodo}[1]{\todo{HOGG}{blue}{#1}}

% Response to referee
\definecolor{mygreen}{rgb}{0, 0.50196, 0}
\newcommand{\response}[1]{#1}
% \newcommand{\response}[1]{{\color{mygreen} {\bf #1}}}

% Document specific variables.
\newcommand{\rate}{\ensuremath{\lambda}}
\newcommand{\intrate}{\ensuremath{\Lambda}}

\begin{document}

\title{%
    Note about the Poisson Process for population inference
}

\newcommand{\uw}{2}
\newcommand{\nyu}{3}
\author{%
    Daniel~Foreman-Mackey\altaffilmark{1,\uw,\nyu}
}
\altaffiltext{1}         {Sagan Fellow; \url{danfm@nyu.edu}}
\altaffiltext{\uw}       {Department of Astronomy, University of Washington,
                          Seattle, WA, 98195, USA}
\altaffiltext{\nyu}      {Center for Cosmology and Particle Physics,
                          Department of Physics, New York University,
                          4 Washington Place, New York, NY, 10003, USA}

\section{The Poisson Process}

Given a rate density function $\rate(\bvec{w})$, the Poisson probability of
observing $N$ events in some volume $\bvec{w} \in \bvec{W}$ is
\begin{eqnarray}\eqlabel{count}
p(N) &=& \frac{\intrate^N\,e^{-\intrate}}{N!}
\end{eqnarray}
where
\begin{eqnarray}
\intrate &=& \int_{\bvec{W}} \rate(\bvec{w}) \dd\bvec{w}
\end{eqnarray}
is the total rate in the volume.

Under this same model, the probability of an event at $\bvec{w}_n$ is
\begin{eqnarray}
p(\bvec{w}_n) &=& \frac{\rate(\bvec{w}_n)}{\intrate} \quad.
\end{eqnarray}
Then, once we have $N$ independent events $\{\bvec{w}_n\}_{n=1}^N$, the
probability across the full sample is
\begin{eqnarray}\eqlabel{ind}
p(\{\bvec{w}_n\}) &=& N!\,\prod_{n=1}^N\frac{\rate(\bvec{w}_n)}{\intrate}
\end{eqnarray}
where the $N!$ combinatoric term comes from the exchangeability of the
samples and it is important for the next step.

Combining \eq{count} and \eq{ind}, we can derive the standard expression for
the Poisson Process likelihood function
\begin{eqnarray}
p(N,\,\{\bvec{w}_n\}) &=& p(N)\,p(\{\bvec{w}_n\}) \\
&=& \frac{\intrate^N\,e^{-\intrate}}{N!} \,
    N!\,\prod_{n=1}^N\frac{\rate(\bvec{w}_n)}{\intrate} \\
&=& e^{-\intrate} \, \prod_{n=1}^N\rate(\bvec{w}_n) \quad.
\end{eqnarray}

\section{Mixture of Poisson Processes}

Let's say that each point $n$ in our sample is generated from one of $K$
distinct populations with underlying rate densities $\rate_k(\bvec{w})$.
In this case, the joint likelihood across the full sample can be written as
\begin{eqnarray}
p(N,\,\{\bvec{w}_n\}\,|\,\{q_n\}) &=&
    \exp\left(-\sum_{k=1}^K\intrate_k\right) \,
    \prod_{n=1}^N\prod_{k=1}^K
        \left[\rate_k(\bvec{w}_n)\right]^{\bvec{1}(q_n = k)}
\end{eqnarray}
where $q_n$ is the class of object $n$ and $\bvec{1}(\cdot)$ is the indicator
function that equals one if the argument is true and zero otherwise.
This equation can be easily derived by splitting the sample into the $K$
subsamples, writing the Poisson Process likelihood for each subsample, and
taking the product.

Now, we will relax our assumptions and allow for the fact that we don't know
the class memberships $\{q_n\}$ \foreign{a priori}.
Instead, we can choose a prior over class memberships
\begin{eqnarray}
p(q_n) &=& \prod_{k=1}^K {Q_k}^{\bvec{1}(q_n = k)}
\end{eqnarray}
where $\sum_k Q_k = 1$, and marginalize over choice of class for each sample.
Doing this, we find
\begin{eqnarray}
p(N,\,\{\bvec{w}_n\}) &=&
    \exp\left(-\sum_{k=1}^K\intrate_k\right) \,
    \prod_{n=1}^N\sum_{k=1}^K
        Q_k\,\rate_k(\bvec{w}_n)
\end{eqnarray}


% \begin{figure}[p]
% \begin{center}
% \includegraphics[width=\textwidth]{figures/smooth/results.pdf}
% \end{center}
% \caption{%
% {\bf Simulated data}.
% Inferences about the rate density based on the simulated catalog \modela.
% \emph{Center:} the points with error bars show the exoplanet candidates in the
% simulated incomplete catalog, the contours show the survey completeness
% function (\citealt{petigura}), and the grayscale shows the median posterior
% occurrence surface.
% \emph{Top and left:} the red dashed line shows the true distribution that was
% used to generate the catalog, the points with error bars show the results of
% the inverse-detection-efficiency procedure, and the histograms are posterior
% samples from the marginalized rate density as inferred by our method.
% \figlabel{smooth-results}}
% \end{figure}

\end{document}
