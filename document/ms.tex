%
%  RULES OF THE GAME
%
%  * 80 characters
%  * line breaks at the ends of sentences
%  * Sun Solar Earth all capitalized when referring to our peeps, even as in
%    "extra-Solar"
%  * eqnarrys ONLY
%  * that is all.
%

\documentclass[12pt,preprint]{aastex}

\pdfoutput=1

\include{vc}

\usepackage{color,hyperref}
\definecolor{linkcolor}{rgb}{0,0,0.5}
\hypersetup{colorlinks=true,linkcolor=linkcolor,citecolor=linkcolor,
            filecolor=linkcolor,urlcolor=linkcolor}
\usepackage{url}
\usepackage{amssymb,amsmath}
\usepackage{subfigure}
\usepackage{booktabs}

\usepackage{natbib}
\bibliographystyle{apj}


\newcommand{\project}[1]{\textsl{#1}}
\newcommand{\kepler}{\project{Kepler}}
\newcommand{\terra}{\project{TERRA}}
\newcommand{\license}{MIT License}

\newcommand{\paper}{\textsl{Article}}

\newcommand{\foreign}[1]{\emph{#1}}
\newcommand{\etal}{\foreign{et\,al.}}
\newcommand{\etc}{\foreign{etc.}}
\newcommand{\True}{\foreign{True}}
\newcommand{\Truth}{\foreign{Truth}}

\newcommand{\figref}[1]{\ref{fig:#1}}
\newcommand{\Fig}[1]{\figurename~\figref{#1}}
\newcommand{\fig}[1]{\Fig{#1}}
\newcommand{\figlabel}[1]{\label{fig:#1}}
\newcommand{\Tab}[1]{Table~\ref{tab:#1}}
\newcommand{\tab}[1]{\Tab{#1}}
\newcommand{\tablabel}[1]{\label{tab:#1}}
\newcommand{\Eq}[1]{Equation~(\ref{eq:#1})}
\newcommand{\eq}[1]{\Eq{#1}}
\newcommand{\eqalt}[1]{Equation~\ref{eq:#1}}
\newcommand{\eqlabel}[1]{\label{eq:#1}}
\newcommand{\sectionname}{Section}
\newcommand{\Sect}[1]{\sectionname~\ref{sect:#1}}
\newcommand{\sect}[1]{\Sect{#1}}
\newcommand{\sectalt}[1]{\ref{sect:#1}}
\newcommand{\App}[1]{Appendix~\ref{sect:#1}}
\newcommand{\app}[1]{\App{#1}}
\newcommand{\sectlabel}[1]{\label{sect:#1}}

\newcommand{\dd}{\ensuremath{\,\mathrm{d}}}
\newcommand{\bvec}[1]{\ensuremath{\boldsymbol{#1}}}
\newcommand{\densityunit}{{\ensuremath{\mathrm{nat}^{-2}}}}

% TO DOS
\newcommand{\todo}[3]{{\color{#2} \emph{#1} TODO: #3}}
\newcommand{\dfmtodo}[1]{\todo{DFM}{red}{#1}}
\newcommand{\hoggtodo}[1]{\todo{HOGG}{blue}{#1}}

% Response to referee
\definecolor{mygreen}{rgb}{0, 0.50196, 0}
\newcommand{\response}[1]{#1}
% \newcommand{\response}[1]{{\color{mygreen} {\bf #1}}}

% Document specific variables.
\newcommand{\rate}{\ensuremath{\Gamma}}
\newcommand{\ratepar}{{\ensuremath{\theta}}}
\newcommand{\ratepars}{{\ensuremath{\bvec{\ratepar}}}}
\newcommand{\obs}[1]{\ensuremath{\hat{#1}}}
\newcommand{\radius}{\ensuremath{R}}
\newcommand{\period}{\ensuremath{P}}
\newcommand{\completeness}{{\ensuremath{Q_\mathrm{c}}}}
\newcommand{\transitprob}{{\ensuremath{Q_\mathrm{t}}}}
\newcommand{\data}{{\ensuremath{\bvec{x}}}}
\newcommand{\entry}{{\ensuremath{\bvec{w}}}}
\newcommand{\catalog}{{\ensuremath{\bvec{\entry}}}}

\newcommand{\interim}{{\ensuremath{\bvec{\alpha}}}}

\newcommand{\binarea}{{\ensuremath{\Delta}}}
\newcommand{\bincenter}{{\ensuremath{\bvec{x}}}}
\newcommand{\binheight}{{\ensuremath{w}}}
\newcommand{\binheights}{{\ensuremath{\bvec{\binheight}}}}

\newcommand{\mean}{{\ensuremath{\mu}}}
\newcommand{\smooth}{{\ensuremath{\lambda}}}
\newcommand{\smoothpars}{{\ensuremath{\bvec{\smooth}}}}
\newcommand{\cov}{{\ensuremath{\mathrm{K}}}}

\newcommand{\gammaearth}{{\ensuremath{\rate_\oplus}}}

\newcommand{\resultsurl}{\url{http://dx.doi.org/10.5281/zenodo.11507}}

\begin{document}

\title{%
    The population of transiting exoplanets
}

\newcommand{\uw}{2}
\newcommand{\nyu}{3}
\newcommand{\mpia}{4}
\newcommand{\cds}{5}
\author{%
    Daniel~Foreman-Mackey\altaffilmark{1,\uw},
    David~W.~Hogg\altaffilmark{\nyu,\mpia,\cds},
    \etal
}
\altaffiltext{1}         {Sagan Fellow; \url{danfm@nyu.edu}}
\altaffiltext{\uw}       {Department of Astronomy, University of Washington,
                          Seattle, WA, 98195, USA}
\altaffiltext{\nyu}      {Center for Cosmology and Particle Physics,
                          Department of Physics, New York University,
                          4 Washington Place, New York, NY, 10003, USA}
\altaffiltext{\mpia}     {Max-Planck-Institut f\"ur Astronomie,
                          K\"onigstuhl 17, D-69117 Heidelberg, Germany}
\altaffiltext{\cds}      {Center for Data Science,
                          New York University,
                          4 Washington Place, New York, NY, 10003, USA}

\begin{abstract}

Blah.

\end{abstract}

\keywords{%
methods: data analysis
---
methods: statistical
---
catalogs
---
planetary systems
---
stars: statistics
}

\section{Introduction}
\sectlabel{intro}

Some words here.

\citet{Foreman-Mackey:2014}

\section{Discussion}

All of the code used in this project is available from
\url{http://github.com/dfm/exopop2} under the MIT open-source software license.
This code (plus some dependencies) can be run to re-generate all of the
figures and results in this \paper; this version of the paper was generated
with git commit \texttt{\githash} (\gitdate).

\acknowledgments
It is a pleasure to thank
\ldots
for helpful contributions to the ideas and code presented here.
% This project was partially supported by the NSF (grant AST-0908357), NASA
% (grant NNX08AJ48G), and the Moore--Sloan Data Science Environment at NYU.
% This research builds on ideas generated at a three-week workshop supported by
% NSF Grant DMS-1127914 to the Statistical and Applied Mathematical Sciences
% Institute.
This research made use of the NASA \project{Astrophysics Data System}.

\bibliography{exopop2}

% \begin{figure}[p]
% \begin{center}
% \includegraphics[width=\textwidth]{figures/smooth/results.pdf}
% \end{center}
% \caption{%
% {\bf Simulated data}.
% Inferences about the rate density based on the simulated catalog \modela.
% \emph{Center:} the points with error bars show the exoplanet candidates in the
% simulated incomplete catalog, the contours show the survey completeness
% function (\citealt{petigura}), and the grayscale shows the median posterior
% occurrence surface.
% \emph{Top and left:} the red dashed line shows the true distribution that was
% used to generate the catalog, the points with error bars show the results of
% the inverse-detection-efficiency procedure, and the histograms are posterior
% samples from the marginalized rate density as inferred by our method.
% \figlabel{smooth-results}}
% \end{figure}

\end{document}
